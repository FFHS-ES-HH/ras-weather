\documentclass[
    10pt,
    a4paper,
    % listof = totoc
]{scrartcl}
\usepackage{ucs}
\usepackage[utf8x]{inputenc}
\usepackage[english,ngerman]{babel}
\selectlanguage{ngerman}
\usepackage[T1]{fontenc}

% Math stuff
\usepackage{amsmath}
\usepackage{amsfonts}
\usepackage{amssymb}

% Farben
\usepackage[usenames,x11names,dvipsnames,rgb]{xcolor}
\definecolor{grey}{rgb}{0.4,0.4,0.4}
\definecolor{lightgrey}{rgb}{0.8,0.8,0.8}
\definecolor{ultralightgrey}{rgb}{0.96,0.96,0.96}

% Grafix
\usepackage{graphicx}

% Schriften
\usepackage{mathpazo,avant,courier}

% TikZ (dot2tex etc.)
% \usepackage{tikz}
% \usetikzlibrary{decorations, arrows, shapes}

% Farben in Tabellen
\usepackage{colortbl}

% Lange Tabellen
\usepackage{longtable}

% Gewrappte boxen (können innerhalb f{rame}box's verwendet werden)
\usepackage{minibox}

% FloatBarrier stellt z.B. sicher, dass das Literaturverzeichnis am Ende des
% Dokuments erscheint.
\usepackage{placeins}

% Hyperref
\usepackage{hyperref}
% Hypersetup
\hypersetup{
    pdftitle = {ES-HH - Software Design ras-weather},
    pdfauthor = {David Daniel},
    pdfsubject = {Software Design ras-weather},
    pdfkeywords = {Software Design} {ES-HH},
    % hidelinks
    colorlinks = true,
    linkcolor = blue,
    % urlcolor = black
    urlcolor = Blue,
    citecolor = grey
}
\urlstyle{same}

% Apa cite style
\usepackage{apacite}

% Glossar (load _after_ ! hyperref)
% \usepackage[toc]{glossaries}
% \makeglossaries
% \newglossaryentry{RTTI}
% {
    % name = {RTTI},
    % description = {"``Run time type information"'' liefert Informationen über
    % benutzerdefinierte Typen zur Laufzeit}
% }

% Listings
% @see http://tex.stackexchange.com/questions/51867/koma-warning-about-toc
% \usepackage{scrhack}
% \usepackage{listings}
% \lstset{
    % breakatwhitespace=true,
    % columns=fullflexible,
    % keepspaces=true,
    % breaklines=true,
    % tabsize=4, 
    % showstringspaces=false,
    % extendedchars=true,
    % basicstyle=\footnotesize\ttfamily,
    % numbers=left,
    % numberstyle=\scriptsize,
    % firstnumber=1
% }
% \lstdefinestyle{custom}{
    % belowcaptionskip=1\baselineskip,
    % captionpos = b,
    % breaklines=true,
    % frame=l,
    % xleftmargin=\parindent,
    % showstringspaces=false,
    % keywordstyle=\bfseries\color{green!40!black},
    % commentstyle=\itshape\color{purple!40!black},
    % identifierstyle=\color{blue},
    % stringstyle=\color{orange},
% }

\title{ES-HH - Software Design}
% \subtitle{}
\author{Andreas Hasler \\{\small andreas.hasler@students.ffhs.ch}
\and David Daniel\\{\small david.daniel@students.ffhs.ch}}
\date{\today}

\begin{document}
\maketitle
\pagenumbering{Alph}% Use uppercase alphabetic page numbers (and reset to A)

\begin{abstract}
    Dieses Dokument erläutert die Architekturüberlegungen zum Software Design für
    ras-weather. Die hier diskutierte Architektur bezieht sich auf die Software, welche
    auf dem Raspberry Pi betrieben wird. Externe Software wie die Web-Applikation oder die
    Smartphone Applikation werden in diesem Dokument nicht beschrieben.
\end{abstract}

\clearpage
\pagenumbering{Roman}% Use uppercase roman page numbers (and reset to I)
\tableofcontents

\section{Analyse}
\pagenumbering{arabic}% Use numeric page numbers (and reset to 1)

Gemäss den Anwendungsfällen aus \cite{project-doc} können die folgenden Subjekte und
Objekte ermittelt werden:

\begin{itemize}
    \item Messwerte (UC 2, 3, 6, 7, 8, 9, 10)
        \begin{itemize}
            \item Luftdruck
            \item Temperatur
            \item Feuchtigkeit
            \item Lichtstärke
        \end{itemize}
\end{itemize}<++>


% \FloatBarrier
% \appendix

% \listoftables
% \listoffigures
% \lstlistoflistings
% \printglossary[title = Glossar, toctitle = Glossar]

% \bibliographystyle{apacite}
% \bibliography{<++>}

% \renewcommand{\refname}{\section{Bibliographie}}
% \begin{thebibliography}{}
    % \bibitem[<+text+>]{<+name+>} <+author+>: <+book+>, <+publisher+>, <+year+>
% \end{thebibliography}<++>

\end{document}
